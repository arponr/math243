\documentclass{amsart}
\title{Simulating Evolutionary Reputation using PageRank}
\author{Arpon Raksit \and Ben Kuhn \and Rahul Dalal}
\date{\today}
\usepackage{enumitem, verbatim}

\newcommand{\lf}{\left}
\newcommand{\ri}{\right}
\newcommand{\md}{\middle}
\newcommand{\mt}{\ri. & \lf.} 
\newcommand{\f}{\frac} 
\newcommand{\df}[2]{{}^{#1} \! /_{#2}} 
\newcommand{\into}{\hookrightarrow}
\newcommand{\onto}{\twoheadrightarrow}
\DeclareMathOperator{\sgn}{sgn}
\DeclareMathOperator{\sig}{sig}
\DeclareMathOperator{\lcm}{lcm}
\DeclareMathOperator{\pf}{pf}
\DeclareMathOperator{\rank}{rank}
\newcommand{\triv}{\mathrm{triv}}
\DeclareMathOperator{\Aut}{Aut}
\DeclareMathOperator{\Gal}{Gal}
\DeclareMathOperator{\Ext}{Ext}
\DeclareMathOperator{\Hom}{Hom}
\DeclareMathOperator{\Sym}{Sym}
\DeclareMathOperator{\vol}{vol}
\DeclareMathOperator{\mesh}{mesh}
\DeclareMathOperator{\ad}{ad}
\DeclareMathOperator{\tr}{tr}
\DeclareMathOperator{\image}{im}
\DeclareMathOperator{\im}{Im}
\DeclareMathOperator{\re}{Re}
\newcommand{\id}{\mathrm{id}}
\DeclareMathOperator{\orb}{orb}
\DeclareMathOperator{\fix}{fix}
\DeclareMathOperator{\stab}{stab}
\newcommand{\herm}[1]{{\overline{#1}^\mathbf T}}
\newcommand{\m}[1]{\mathbf{#1}}
\newcommand{\td}[1]{\tilde{#1}}
\newcommand{\Hlt}{\mathbb H}
\newcommand{\C}{\mathbb C}
\newcommand{\R}{\mathbb R}
\newcommand{\Q}{\mathbb Q}
\newcommand{\N}{\mathbb N}
\newcommand{\Z}{\mathbb Z}
\newcommand{\T}{\mathbb T}
\newcommand{\F}{\mathbb F}
\newcommand{\RP}{\mathbb{RP}}
\newcommand{\CP}{\mathbb{CP}}
\newcommand{\eps}{\epsilon}
\newcommand{\om}{\omega}
\newcommand{\pdif}[2]{\f{\partial #1}{\partial #2}}
\newcommand{\dif}[2]{\f{d#1}{d#2}}
\newcommand{\bt}{\bullet}

\begin{document}

\begin{abstract}
A key question in the study of evolutionary dynamics has been how indirect reciprocity might emerge: that is, how individuals might evolve a tendency to accept a personal fitness cost in exchange for a greater benefit to another individual, without the understanding that the recipient will later repay them. One proposed mechanism for this to occur is that donating to others enhances one's reputation, making third parties more likely to donate to the donator. We develop a novel model for determining reputation, adapted from the PageRank Web search algorithm, that allows individuals to differentiate between refusal to donate because the donor is selfish, and refusing to donate because the recipient is selfish. Thus donors will be penalized less for being paired against a number of selfish individuals. Using this model, we find that cooperation can emerge with much smaller benefit/cost ratios than previously observed.
\end{abstract}

\maketitle

\section{Introduction}

\section{Related Work}

Previously, Nowak and Sigmund have modeled indirect reciprocity with a concept of ``image'' \cite{nowak_evolution_1998}. Each individual in the simulation is given an image score. During each round of a stochastic simulation, individuals play a number of single-round donor-receiver games with other members of the population and have the opportunity to be either altruistic or selfish; the former increases the image score and the latter decreases it. Individuals’ strategies are to donate if the recipient’s image score is above some threshold, but keep the reward otherwise.

Under these circumstances, Nowak and Sigmund found that in the simplest model, without mutation, whether cooperation evolves is determined by the initial fraction of defectors. Adding mutation causes the population to cycle between discriminating cooperation, unconditional cooperation, and defection in all cases. Limiting the number of interactions known by each player makes it difficult to establish cooperation in large populations. Finally, if players are conscious of their own reputation and act to increase it when it is low, cooperation evolves much more easily. 

This model is simple and elegant, but it has a few drawbacks. For instance, it discretizes reputation, which may lead to strange edge effects. Furthermore, it penalizes equally an individual who defects because they are playing against a low-reputation opponent, and one who defects for selfish reasons. Finally, a model based on per-pair reputation allows for later considering strategies that involve lying about reputation, allowing us to potentially simulate more complex social dynamics.

\section{Our model}

\subsection{Parameters}

Our model comprises a series of rounds in which a number of donor-receiver games are played. We use the following parameters and variables:

\begin{itemize}
\item $N \in \mathbb{N}$ is the size of the population to be tested.
\item $M \in \mathcal{M}_N(\{0,1\})$ is an $N \times N$ matrix representing the opinions each individual has of the others; $M_{ij}$ is 1 if player $i$ donated to $j$ the last time they had the chance and 0 if $i$ was selfish. Initially, $M$ is the zero matrix.
\item $I$ is the number of donor-receiver interactions per round.
\item $b$ is the benefit-cost ratio (fitness benefit to recipient over cost to donor).
\item $\omega$ is the selection coefficient (magnitude of cost to donor).
\item $f \in [0, \infty)^N$ is a vector of fitnesses in the current round; at the end of the round, individuals reproduce with probability proportional to their fitness.
\item $r \in [0,1]^N$ is a vector of reputations of individuals.
\item $s \in [0,1]^N$ is a vector of strategies; if $s_i = x$, then player $i$ will donate to any players with $r_i \ge x$ and keep the reward against other players.
\end{itemize}

\subsection{Round protocol}

Each round comprises the following steps:

\begin{enumerate}
\item Set all $f_i$ to 1.
\item Compute $r_i$ according to the PageRank algorithm (see below).
\item Determine the results of $I$ interactions (see below).
\item Set entries of $f$ according to the results of the interactions.
\item Set entries of $M$ according to the results of the interactions.
\item Choose a player to die at random and one to reproduce proportional to fitness. Replace the dead player with a copy of the reproducing one (It inherits $M$ entries).
\end{enumerate}

\subsection{PageRank}
\newcommand{\tM}{\tilde M}
We think of the matrix $M$ as a directed graph with some $M_{ij} = 1$ representing an edge from $j$ to $i$---in other words $j$ has endorsed $i$. Noting an analogy between these endorsements and links between web pages, we run the standard page rank algorithm on this graph.

Given a matrix $M$, we compute the PageRank as follows. Let $\tM$ be the column-normalization of $M$, so that $\tM_i = \frac{M_i}{\sum_j M_{ij}}$. (We need a column-normalized matrix to run PageRank). The idea of PageRank is to treat $\tM$ as a Markov chain and find its stationary probability. The Perron-Frobenius theorem says that $\tM$ has at a unique non-negative eigenvector (and therefore a unique PageRank vector) if and only if it is irreducible. But we are not necessarily guaranteed that this is the case (for example, if an unconditional cooperator donates to an unconditional defector, then the chain is reducible). Therefore, we add some ``teleportation probability'' $\alpha$ such that from any given state there is a probability $1-\alpha$ of going to a random state.

We compute PageRank using the iterated multiplication method; that is, let $u$ be a uniform vector, $r_0 = u$, and let $r_i = (1-\alpha)\tilde M r_{i-1} + \alpha u$. Once the difference between successive $r_i$ is small enough, the last computed vector is the PageRank.

We make one modification to the PageRank algorithm. Because $r$ is normalized to the sum of the entries being $1$, this algorithm does not discriminate between the entire population cooperating and the entire population defecting. Therefore, we normalize to the sum being the average number of cooperations per player.

\subsection{Interaction protocol}
$I$ times two separate players $i,j$ are chosen independently of all other choices (note that this means that in a given round, the same pair can be chosen more than once). $i$ has a choice to decrease its own fitness by $\omega$ to increase that of the $j$ by $b \omega$. It chooses according to its strategy doing so iff $s_i < r_j$. If it did, $M_{ij}$ is set to one, otherwise $M_{ij}$ is set to $0$. 

There were two modifications we made to this protocol in some tests. First there was a probability $\alpha$ that $i$ would do the opposite of what it wanted to. Second, there was a probability $\beta$ that even if $i$ donated to $j$, $M_{ij}$ may not have been set to $0$. 

\section{Results}

\section{Conclusion}

\section{Code}
\begin{verbatim}
[Insert Code Here]
\end{verbatim}


\bibliographystyle{plain}
\bibliography{refs}

\end{document}
