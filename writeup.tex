\documentclass{amsart}
\title{Simulating Evolutionary Reputation using PageRank}
\author{Arpon Raksit \and Ben Kuhn \and Rahul Dalal}
\date{\today}
\usepackage{enumitem}

\newcommand{\lf}{\left}
\newcommand{\ri}{\right}
\newcommand{\eps}{\epsilon}
\newcommand{\om}{\omega}

\begin{document}

\begin{abstract}
A key question in the study of evolutionary dynamics has been how
indirect reciprocity might emerge: that is, how individuals might
evolve a tendency to accept a personal fitness cost in exchange for a
greater benefit to another individual, without the understanding that
the recipient will later repay them. One proposed mechanism for this
to occur is that donating to others enhances one's reputation, making
third parties more likely to donate to the donator. We develop a novel
model for determining reputation, adapted from the PageRank Web search
algorithm, that allows individuals to differentiate between refusal to
donate because the donor is selfish, and refusing to donate because
the recipient is selfish. Thus donors will be penalized less for being
paired against a number of selfish individuals. Using this model, we
find that cooperation can emerge with much smaller benefit/cost ratios
than previously observed.
\end{abstract}

\maketitle

\section{Introduction}

In studying the evolutionary origins of altruistic behavior, one of
the most common mechanisms considered is that of reciprocity: an
individual accepts a cost to itself in order to grant a greater
benefit to another. In the most intuitive versions of reciprocity, the
payback (in the form of a reciprocal donation) is immediate; that is,
the players are playing some iterated game where non-cooperation can
immediately be punished. However, ``in the wild'' we often observe
even more altruistic behavior, in which an individual sacrifices some
fitness even when it is not guaranteed that the beneficiary will ever
have the ability to reciprocate, under the assumption that other
members of the population will do a similar service to the donor. This
type of interaction is termed ``indirect reciprocity'' and there are
several proposed mechanisms for it.

We consider the mechanism of reputation: when an individual chooses to
donate, it enhances their reputation in the population, and
individuals are more likely to donate to individuals with higher
reputation. Reputation models sometimes suffer from several
problems. In many models, individuals are penalized equally for being
selfish no matter the reputation of their opponent, so that if an
otherwise generous player has the bad luck to be paired against
several sociopaths, their reputation will be penalized unfairly. Also,
Leimar and Hammerstein \cite{leimar_evolution_2001} argue that in many
reputational models, the dominant strategy is to consider only one's
own reputation and not the recipients, and allowing these strategies
threatens the evolutionary stability of cooperative ones. We try to
find a model that limits these effects.

In this paper, we adapt the PageRank algorithm from Web searching to
give a reputation score to each individual based on the matrix of
pairwise opinions of each individual about each other. These opinions
are based on the last interaction between the two. We give a
stochastic model for the evolutionary dynamics of a population
undergoing reputation interactions and give some arguments as to why
this model is realistic and avoids the problems mentioned above.
%TODO

Using this model, we show that indirect reciprocity can emerge with a
much lower benefit/cost ratio than previously shown. We show that
altruism can spontaneously emerge and fixate even in large populations
of reasonably selfish individuals.
%TODO based on experimental results

The remainder of the paper is as follows: in section
\ref{sec:related}, we go over previous work and explain our novel
contributions. In section \ref{sec:model}, we describe our model and
explain our parameter choices. In section \ref{sec:results}, we
describe and analyze the results of our simulations. Finally, in
section \ref{sec:conclusion} we draw conclusions and suggest further
areas of research.

\section{Related Work}\label{sec:related}

Previously, Nowak and Sigmund have modeled indirect reciprocity with a
concept of ``image'' \cite{nowak_evolution_1998}. Each individual in
the simulation is given an image score. During each round of a
stochastic simulation, individuals play a number of single-round
donor-receiver games with other members of the population and have the
opportunity to be either altruistic or selfish; the former increases
the image score and the latter decreases it. Individuals’ strategies
are to donate if the recipient’s image score is above some threshold,
but keep the reward otherwise.

Under these circumstances, Nowak and Sigmund found that in the
simplest model, without mutation, whether cooperation evolves is
determined by the initial fraction of defectors. Adding mutation
causes the population to cycle between discriminating cooperation,
unconditional cooperation, and defection in all cases. Limiting the
number of interactions known by each player makes it difficult to
establish cooperation in large populations. Finally, if players are
conscious of their own reputation and act to increase it when it is
low, cooperation evolves much more easily.

This model is simple and elegant, but it has a few drawbacks. For
instance, it discretizes reputation, which may lead to strange edge
effects. Furthermore, it penalizes equally an individual who defects
because they are playing against a low-reputation opponent, and one
who defects for selfish reasons. Finally, a model based on per-pair
reputation allows for later considering strategies that involve lying
about reputation, allowing us to potentially simulate more complex
social dynamics.

\section{Our model}
\label{sec:model}

\subsection{Parameters}

The basic model comprises a series of rounds in which a number of
donor-receiver games are played. We use the following parameters:

\begin{itemize}
\item $N$ is the size of the population to be tested.
\item $I$ is the number of donor-receiver interactions per round.
\item $b$ is the benefit-cost ratio (fitness benefit to recipient over
  cost to donor).
\item $\omega$ is the selection coefficient (magnitude of cost to
  donor).
\item $\mu$ is the strategy mutation probability.
\item $c$ is the fraction of players that are unconditional.
  cooperators
\item $d$ is the fraction that are unconditional defectors.
\end{itemize}

\subsection{State}

Our model has the following state variables:

\begin{itemize}
\item $M \in \mathcal{M}_N(\{0,1\})$ is an $N \times N$ matrix
  representing the opinions each individual has of the others;
  $M_{ij}$ is 1 if player $i$ donated to $j$ the last time they had
  the chance and 0 if $i$ was selfish. Initially, $M$ is the zero
  matrix.
\item $f \in [0, \infty)^N$ is a vector of fitnesses in the current
  round; at the end of the round, individuals reproduce with
  probability proportional to their fitness.
\item $r \in [0,1]^N$ is a vector of reputations of individuals.
\item $s \in [0,1]^N$ is a vector of strategies; player $i$ will
  donate to any player $j$ with $r_j > s_i$ and keep the reward
  against other players.
\end{itemize}

\subsection{Round protocol}
The simulation is initialized by giving each player a random strategy:
unconditional cooperation with probability $c$, unconditional
defection with probability $d$, and a uniform $s \in [0,1]$ with
probability $1 - c- d$. $M$ is originally set to be $0$.

Each round comprises the following steps:

\begin{enumerate}
\item Set all $f_i$ to 1.
\item Compute $r_i$ according to the PageRank algorithm (see below).
\item Determine the results of $I$ interactions (see below).
\item Set entries of $f$ according to the results of the interactions.
\item Set entries of $M$ according to the results of the interactions.
\item Choose a player to die at random and one to reproduce
  proportional to fitness. Replace the dead player with a copy of the
  reproducing one (inheriting the appropriate row and column of $M$).
\item With probability $\mu$, the strategy of the new player resets
  randomly as in the initialization phase.
\end{enumerate}

\subsection{PageRank}
\newcommand{\tM}{\tilde M} We think of the matrix $M$ as a directed
graph with some $M_{ij} = 1$ representing an edge from $j$ to $i$---in
other words $j$ has endorsed $i$. Noting an analogy between these
endorsements and links between web pages, we run the standard PageRank
algorithm \cite{page_pagerank_1999} on this graph. (See section
\ref{sec:whypagerank} for an explanation of why this is a realistic
algorithm for individuals to employ.)

Given a matrix $M$, we compute the PageRank as follows. Let $\tM$ be
the column-normalization of $M$, so that $\tM_i = \frac{M_i}{\sum_j
  M_{ij}}$. (We need a column-normalized matrix to run PageRank, since
it is based on interpreting the matrix as a Markov chain). The idea
of PageRank is to treat $\tM$ as a Markov chain and find its
stationary probability. The Perron-Frobenius theorem says that $\tM$
has a unique non-negative eigenvector (and therefore a unique PageRank
vector) if and only if it is irreducible. But we are not necessarily
guaranteed that this is the case: there easily may be a player who has
not been cooperated with. Therefore, we add some ``teleportation
probability'' $\alpha$ such that from any given state there is a
probability $1-\alpha$ of going to a random state.

We compute PageRank using the iterated multiplication method; that is,
let $u$ be a uniform vector, $r_0 = u$, and let $r_i =
(1-\alpha)\tilde M r_{i-1} + \alpha u$. Once the difference between
successive $r_i$ is small enough, the last computed vector is the
PageRank.

We make one modification to the PageRank algorithm. Because $r$ is
normalized to the sum of the entries being $1$, this algorithm does
not discriminate between the entire population cooperating and the
entire population defecting. Therefore, at the end we multiply $r$ by
$C/(N-1)$, where $C = \sum_{i,j} M_{ij}$ is the sum of the entries in
the opinion matrix (a measure of how cooperative the current
population is). Note that $0 \le C \le N(N-1)$, the lower bound
occurring if all players are defecting and the upper bound if everyone
is cooperating. Hence the maximum of each $r_i$ is $1$.

\subsection{Interaction protocol}
In each round $I$ ordered pairs $(i,j)$ of distinct players are
chosen, each pair independent of the others (note that this means that
in a given round, the same pair can be chosen more than once). Each
$i$ has a choice to decrease its own fitness by $\omega$ to increase
the fitness of $j$ by $b \omega$. The choice is made according to the
strategy of $i$: cooperation occurs if and only if $s_i < r_j$. If
cooperation occurs, $M_{ij}$ is set to one, otherwise $M_{ij}$ is set
to $0$.

\subsection{Modifications}
We tested a few modifications of the original model
\begin{enumerate}
%please add or remove as necessary
\item
There is a small probability $\alpha$ that a player does the opposite
of what it wants to in an interaction
\item
There is a small probability $\beta$ that after a donation, the
corresponding entry of $M$ is not updated.
\item
Each player computes a different number of steps $n_i$ of the PageRank
iterated multiplication representing different capacities for analysis
of social situations. For example, someone who just does $1$ step
would measure reputation as the total number of donations a player
gave replicating Nowak and Sigmund's original model. The $n_i$ are
initialized randomly, are inherited, and mutate with some probability
$\mu_n$. There is a fitness cost $p n_i$ to each player $i$,
reflecting the evolutionary disadvantage from devoting more resources
to social computation.
\item
When an $M_{ij}$ entry is updated a weighted average with some weight
$m$ is taken with the new value and the old value representing some
memory of past interactions.
\end{enumerate}

\section{Analysis of model}
\label{sec:analysis}

\subsection{Use of PageRank}
\label{sec:whypagerank}

The PageRank algorithm is frequently used to calculate reputation in
social networks \cite{pujol_extracting_2002}. There are a few features
of PageRank that make it desirable for our simulation:
\begin{enumerate}
\item
The opinions of more generous players matter more. This is because
more generous players have more endorsements pointing to them, and
thus better reputation; therefore an endorsement from a high-raking
individual confers more reputation on the receiver than a reputation
from a lower individual. For this reason, optimal strategies must take
into account the receiver's reputation as well as the donor's,
resolving Leimar and Hammerstein's complaint
\cite{leimar_evolution_2001} that reputation dynamics do not properly
incentivize players.
\item
Conversely, the effects of selfishness against low-ranked individuals
are minimized: low-reputation groups cannot unilaterally affect the
PageRank graph by a large margin
\cite{langville_deeper_2004}. Therefore, as mentioned above, generous
individuals will not be unduly penalized for encountering a number of
selfish individuals and refusing to cooperate.
\end{enumerate}

\subsection{Interactions per round}

While it is customary to have every member of the population interact
every round of the simulation, we chose instead to pick random pairs
of members interact. This is because if two players are assured of
encountering each other again in the future, the opinion
matrix/PageRank system could essentially act as a proxy for a direct
reciprocity calculation. To make sure that this effect was mitigated,
we chose values for $I$ that ensured the two interacting organisms
were not likely to encounter one another again before they died.

Specifically, in a population of $N$, since one player dies every
round, each player dies with probability $\frac1N$ and so a player's
lifespan is a geometrically distributed random variable with mean
$N$. If we pick $I=N$, then on any given round and for any given
players $A$ and $B$, player $A$ interacts with $B$ with probability
$\frac1N$. Therefore if $I\le N$ we expect on average one interaction
between any two given players before one dies. This should be enough
to minimize the effects of our model proxying direct reciprocity.

\section{Results}
\label{sec:results}

\section{Conclusion}
\label{sec:conclusion}

\bibliographystyle{plain} \bibliography{refs}

\end{document}
